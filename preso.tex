%% Простая презентация с примером включения программного кода и
%% пошаговых спецэффектов
\documentclass{beamer}
\usepackage{fontspec}
\usepackage{xunicode}
\usepackage{xltxtra}
\usepackage{xecyr}
\usepackage{hyperref}
\setmainfont{CMU Serif}                 
\setsansfont{CMU Sans Serif}            
\setmonofont{CMU Typewriter Text}
\usepackage{polyglossia}
\setdefaultlanguage{russian}
\usepackage{graphicx}
\usepackage{listings}
\lstdefinestyle{mycode}{
  belowcaptionskip=1\baselineskip,
  breaklines=true,
  xleftmargin=\parindent,
  showstringspaces=false,
  basicstyle=\footnotesize\ttfamily,
  keywordstyle=\bfseries,
  commentstyle=\itshape\color{gray!40!black},
  stringstyle=\color{red},
  numbers=left,
  numbersep=5pt,
  numberstyle=\tiny\color{gray},
}
\lstset{escapechar=@,style=mycode}

\begin{document}
\title{Частотный критерий отсутствия периодических режимов для задачи управления}
\subtitle{предварительные результаты}
\author{Астахов Марк, Давыдов Алексей\\{\footnotesize\textcolor{gray}{группа 424\\руководитель Г.А. Леонов}}}
\institute{СПбГУ}
\frame{\titlepage}

\begin{frame}\frametitle{Введение}
Расcматривается система 
$$ \left\{
\begin{array}{l}
\frac{dx}{dt} = Ax + b\varphi(\sigma),\\
\sigma = c^{*}x,\\
\end{array}
\right. \eqno{(1)}$$
где $A$ -- постоянная $n \times n$ -- матрица, $b$ и $c$ -- постоянные $n$-мерные векторы, $\varphi(\sigma)$ - непрерывная функция.

\bigskip
Пусть дополнительно выполнено условие: $ 0 \leq \frac{\varphi}{\sigma} \leq M $ для всех $\sigma \neq 0$.

Следовательно, отсюда $$\varphi (\sigma - \frac{\varphi}{M}) \geq 0. \eqno{(2)}$$
\end{frame}


\begin{frame}[fragile]\frametitle{}

Разложим $\sigma(t)$ и $\varphi(t)$ в ряд Фурье по ортогональной системе функций Уолша:

    $$\sigma(t) = {\sum\limits^{\infty}_{k=0} {a_{k}{wal}_{k}(t/T)}, \mbox{ }\varphi(t) = {\sum\limits^{\infty}_{k=0} {b_{k}{wal}_{k}(t/T)}, \mbox{ }\mbox{ }\mbox{ }\eqno{(3.1)}$$

где $$a_k = \frac{1}{T}\int\limits_0^T \sigma(t){wal}_{k}(t/T)\,dt, \mbox{ } b_k = \frac{1}{T}\int\limits_0^T \varphi(t){wal}_{k}(t/T)\,dt, \eqno{(3.2)}$$

\bigskip
${wal}_{k}(t) = sign(sin(\pi kt))$ - $k$-я функция Уолша.
\end{frame}

\begin{frame}\frametitle{}
Далее, для системы (1) верно соотношение: $$\sigma(t) = \alpha(t) + \int\limits_0^t \gamma(t-\tau)\varphi(\tau)\,d\tau, \eqno{(4)}$$

где $\alpha(t) = c^{*}e^{At}x_{0}, \gamma(t) = c^{*}e^{At}b, x(0) = x_0 = 0$.

Подставив (4) в формулу  (3.2) для $a_k$ иммем: $$ a_k = \frac{1}{T}\int\limits_0^T (\int\limits_0^t \gamma(t-\tau)\varphi(\tau)\,d\tau) {wal}_{k}(t/T)\,dt. $$


\end{frame}

\begin{frame}\frametitle{}
Окончательно, перепишем неравенство (2), используя представление функций $\varphi(t)$ и $\sigma(t)$ в виде ряда Фурье (3.1), проинтегрируем по периоду T и получим: 
$$ -{\sum\limits^{\infty}_{k=0} {(\frac{1}{M} - \frac{a_k}{b_k}) b_k^2}}\geq 0.$$

Таким образом, периодический режим может существовать лишь в том случае, если хотя один из коэффициентов при $b_k^2$ отрицателен. Следовательно, критерий отсутствия периодических режимов имеет вид: $$ \frac{1}{M} - \frac{a_k}{b_k} \geq 0, k = 0,1,\ldots $$



\end{frame}

\begin{frame}\frametitle{План}
Далее предлагается численно проверить полученный критерий для разных передаточных функций $W(p)$ и сравнить с результат с результатом, где в качестве в $\varphi$ и $\sigma$ выступает тригонометрический ряд Фурье.

\end{frame}


\begin{frame}\frametitle{Список литературы}

1.{\it Леонов Г.A.} Теория управления. - СПб.: Изд-во C.-Петерб. ун-та, 2006. - 233 c.

2.{\it Гарбер Е.Д.} О частотных критериях отсутствия периодических режимов, Автомат. и телемех., 1967, № 11, 178–182.

3.{\it Голубов Б. И., Ефимов А. В., Скворцов В. А.} Ряды и преобразования Уолша: теория и применения. — М.: Изд-во Наука, 1987. - 344 с.

\end{frame}


\end{document}